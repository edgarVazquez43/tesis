\documentclass[../tesis_main.tex]{subfiles}

\chapter*{}

\raggedleft
\section*{Agradezco a ...}

\begin{flushright}

\vspace{0.7in}

...a mi familia, por su total cariño, formación y apoyo,\\
\vspace{0.3in}
...a mi madre por creer siempre en mí,\\
\vspace{0.3in}
...a mi padre por su sacrificio,\\
\vspace{0.3in}
...a todos y cada uno de los integrantes del laboratorio\\
 de Biorobótica por sus múltiples enseñanzas,\\
\vspace{0.3in}
...al Dr. Jesús Savage Carmona, por permitirme\\
ser parte de su equipo de trabajo,\\
\vspace{0.3in}
...al Mtro. Marco Antonio Negrete Villanueva,\\
por su paciencia y apoyo en todo momento,\\
\vspace{0.3in}
...a Jaqui por ser mi apoyo y escucha en momentos inciertos,\\
\vspace{0.3in}
...a Feliza por ser una excelente consejera y amiga,\\
\vspace{0.3in}
...a Bere por el tiempo y esperanzas depositados en mí,\\

\vspace{1.0in}

\textit{Se agradece al CONACYT, a través del proyecto 245491, "Laboratorio de Movilidad e Infraestructura Verde para la Eficiencia Energética en Ciudades", por el apoyo recibido en la realización de este documento.}

\end{flushright}



\raggedright
\newpage{}
\vspace{0.7in}

\chapter*{}
\section*{Resumen}
	El reconocimiento de objetos y la adecuada manipulación de los mismos es una problemática común en el área de la robótica de servicios. El presente documento aborda el diseño de un sistema de manipulación de objetos formado por un brazo robótico de 7DOF, el desarrollo de un algoritmo de visión computacional para obtener la posición y orientación de los objetos, la puesta a prueba de diferentes algoritmos para el cálculo de la cinemática inversa del brazo robótico y finalmente la planeación de acciones entre la detección de un objeto y su correcta manipulación.\\

	Los algoritmos desarrollados en el presente documento puden ser utilizados en la segmetación de objetos de manera general, y podrán ser implementados para realizar un seguimiento de peatones o usuarios de bicicletas en un sistema de monitoreo para bicicletas ecologicas como parte del proyecto "Laboratorio de Movilidad e Infraestructura Verde para la Eficiencia Energética en Ciudades".\\