\documentclass[../tesis_main.tex]{subfiles}

\chapter*{}
\section*{Agradecimientos}

\begin{flushright}
\vspace{0.7in}

\vspace{0.7in}

A mi familia,\\
\vspace{0.3in}
al laboratorio de Biorobótica,\\
\vspace{0.3in}
al Dr. Jesús Savage Carmona,\\
\vspace{0.3in}
al Mtro. Marco Antonio Negrete Villanueva,\\
\vspace{0.3in}
por su paciencia y apoyo en todo momento.\\


\vspace{1.0in}

\textit{Se agradece al CONACYT, a través del proyecto 245491, "Laboratorio de Movilidad e Infraestructura Verde para la Eficiencia Energética en Ciudades", por el apoyo recibido en la realización de este documento.}

\end{flushright}




\newpage{}
\vspace{0.7in}

\chapter*{}
\section*{Abstract}
	El reconocimiento de objetos y la adecuada manipulación de los mismos es una problemática común en el área de la robótica de servicios. El presente documento aborda el diseño de un sistema de manipulación de objetos formado por un brazo robótico de 7DOF, el desarrollo de un algoritmo de visión computacional para reconocer la posición y orientación de los objetos, el desarrollo de la cinemática inversa del brazo robótico y finalmente la planeación de acciones entre la detección de un objeto y su correcta manipulación.\\

	Los algoritmos desarrollados en el presente documento puden ser utilizados en la segmetación de objetos de manera general, y podrán ser implementados para realizar un seguimiento de peatones o usuarios de bicicletas en un sistema de monitoreo para bicicletas ecologicas como parte del proyecto "Laboratorio de Movilidad e Infraestructura Verde para la Eficiencia Energética en Ciudades".\\