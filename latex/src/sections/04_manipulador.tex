\documentclass[../tesis_main.text]{subfiles}


\chapter{Características del manipulador de 7DOF}
	En esta sección se hablará de las características mecánicas del brazo robótico ensamblado en el robot de servicio Justina. Se describirán los elementos que lo componen haciendo énfasis en la ubicación de los actuadores y cómo influye esto en el modelo matemático del mismo. Posteriormente se abordarán los temas respectivos a la cinemática directa e inversa del manipulador antes mencionado.\\


%%%%%%%%%%%%%%%%%%%%%%%%%%%%%%%%%%%%%%%%%%%%%%%%%%%%%%%%%%%%%%%%
           %%%%%%  DESCRPCIÓN SERVOMOTORES   %%%%%%%%
%%%%%%%%%%%%%%%%%%%%%%%%%%%%%%%%%%%%%%%%%%%%%%%%%%%%%%%%%%%%%%%%
	\section{Descripción del hardware}
		Los servomotores Dynamixel son un sistema de actuador inteligente desarrollados con el propósito de funcionar como uniones en un robot o estructura mecánica. Los servomotores Dynamixel están diseñados para ser modulares y soportar la conexión en cadena en cualquier robot o diseño mecánico. Este tipo de servomotores es popular por los beneficios que ofrece: movimientos potentes y flexibles \cite{mensink2008characterization}.\\

		El conjunto de servomotores Dynamixel son un grupo de actuadores de alto rendimiento con reductor, controlador y un protocolo de comunicación completamente integrados. El estado del actuador se puede leer y monitorear a través de un flujo de paquetes de datos \cite{dynamixelEpage}.\\

			\subsection{Motores dynamixel Serie MX}
		Los motores Dynamixel de la serie MX cuentan con la implementación de un algoritmo de control PID para mantener la posición del eje. Las ganancias del algoritmo de control PID se pueden ajustar individualmente para cada servo, lo que le permite controlar la velocidad y par del motor. Todos los servos de la serie MX usan un voltaje nominal de 12v.\\

		En lo que respecta a la parte mecánica, todos los servos Dynamixel son compatibles con una amplia variedad de bridas, acoplamientos y sujetadores lo cual facilita la construcción de cualquier configuración deseada. Este conjunto de elementos mecánicos permite una conexión directa de cualquier modelo de servomotores Dynamixel, lo que proporciona una gran variedad de configuraciones para cualquier necesidad. Dada la gran variedad de configuraciones, este tipo de elementos de unión y fijación resultó de gran ayuda mecánica al momento de construir el manipulador donde se implementaron los algoritmos descritos en el presente trabajo.\\

		\begin{figure}[h!]
			\begin{center}
			\includegraphics[width=12.5cm, height=5.0cm]{manipulator/dynamixel_conexions.jpg}
			\caption{Variedad de ensambles para servomotores.}
			\end{center}
		\end{figure}

		Resumiendo las características de los sermovomotores de la serie MX, podemos enlistar las siguientes:\\ 

		\begin{itemize}
			\item{Detección de posición sin contacto debido a la implementación del sensor de efecto hall.}
			\item{Comunicación de alta velocidad de hasta 4.5 Mbps.}
			\item{Control PID para autocorrección en posicionamiento. [4096 steps] }
			\item{Control de par basado en corriente (2.69mA/step)}
		\end{itemize}

		Si se desea controlar este servo Dynamixel desde un equipo de computo es necesario contar con un dispositivo que facilite la comunicación entre este y el microprocesador de cada uno de los servomotores. El dispositivo que cumple con esta funcionalidad es el USB2Dynamixel, utilizado en la implementación de este trabajo.\\


		\begin{figure}[h!]
			\begin{center}
			\includegraphics[width=9.0cm, height=6.5cm]{manipulator/usb2dynamixel.jpg}	
			\caption{Dispositivo de comunicación USB a protocolo serial.}
			\end{center}
		\end{figure}

		Los motores que componen esta gama comparten características, las cuales se mencionan a continuación:\\

\begin{table}[h!]
\centering 
\begin{tabular}{ |p{6.5cm}||p{2.1cm}|p{2.1cm}|p{2.1cm}|  }
 \hline
 \multicolumn{4}{|c|}{\textbf{MOTORES GAMA MX} } \\
 \hline
 Voltaje de operación  &  14.8v   &	   12v   &   11.1v\\
 \hline
 Protocolo             &  \multicolumn{3}{|c|}{TTL Serial Asincrono} \\
 \hline
 Velocidad del puerto  &  \multicolumn{3}{|c|}{8000bps - 3Mbps} \\
 \hline
 Retroalimentación de posición & \multicolumn{3}{|c|}{ SI} \\
 \hline
 Retroalimentación de temperatura & \multicolumn{3}{|c|}{ SI}\\
 \hline
 Control PID           & \multicolumn{3}{|c|}{ SI } \\
 \hline
 Material              & \multicolumn{3}{|c|}{ Carcasa plástica y reducción metálica} \\
 \hline
 Motor                 & \multicolumn{3}{|c|}{ Maxon RE-MAX} \\
 \hline
\end{tabular}
\caption{Características sobresalientes de los servomotores dynamixel de gamma MX.}
\label{t_manip:1}
\end{table}



\newpage
%%%%%%%%%%%%%%%%%%%%%%%%%%%%%%%%%%%%%%%%%%%%%%%%%%%%%%%%%%%%%%%%%%
		\subsection{Características de los servomotores dynamixel MX-106}
%%%%%%%%%%%%%%%%%%%%%%%%%%%%%%%%%%%%%%%%%%%%%%%%%%%%%%%%%%%%%%%%%%

\begin{table}[h]
\centering
\begin{tabular}{ |p{6.5cm}||p{2.1cm}|p{2.1cm}|p{2.1cm}|  }
 \hline
 \multicolumn{4}{|c|}{\textbf{12 volts} } \\
 \hline
 Voltaje de operación  &  MX-106    &   MX-64   &  MX-28\\
 \hline
 Par de bloqueo        &  8.4[N.m] (5.2[A])  &   6.0[N.m] (4.1[A])  &  2.5[N.m] (1.4[A])\\
 Velocidad sin carga   &  45rpm     &  63rpm    &  55rpm\\
 \hline
 Peso                  &  153g      &  126g     &  72g\\
 \hline
 Resolución            &  \multicolumn{3}{|c|}{ 0.088 grado/valor de registro} \\
 \hline
 Relación de reducción &  \multicolumn{3}{|c|}{1/200} \\
 \hline
 Corriente máxima      &  5.2A@12V & 4.1A@12V & 1.4A@12V\\
 \hline
\end{tabular}
\caption{Características dynamixel modelo MX-106 \cite{mensink2008characterization}.}
\label{t_manip:2}
\end{table}


\begin{figure}[H]
	\begin{center}
	\includegraphics[width=5.5cm, height=5.5cm]{manipulator/MX_106R_2.jpg}	
	\caption{Servomotor dynamixel modelo MX-106.}
	\end{center}
\end{figure}

	La importancia de conocer las características del servomotor MX-106 radican en poder determinar si las capacidades del actuador son suficientes para la aplicación debida. En la configuración del manipulador se tienen dos motores del modelo antes mencionado en un acoplamiento paralelo compartiendo el mismo eje de acción con lo cual se tiene la suma del par de torsión en ese eje. La suma total del par proporcionado por el arreglo de servomotores Dynamixel MX-106 es de $16.8[N.m]$.\\

	Se realizó la medición de la masa total de manipulador, siendo esta de $1.8[kg]$ y por tanto\\

	\begin{equation}
	\begin{aligned}
	\begin{split}
		p & = m g \cr
		  & = 1.8 [kg]\; 9.81 [m / s^2] \cr
		  & = 17.658 [N]
    \end{split}
    \end{aligned}
    \end{equation}


	se sabe que un par de fuerzas está caracterizado por el momento que estas generan:\\ 
	
	\begin{equation}
	\begin{aligned}
	\begin{split}
		M = F_1 \; d = F_2 \;  d
    \end{split}
    \end{aligned}
    \end{equation}

    Se toma el caso límite donde suponemos el peso ejercido como una carga concentrada en el extremo del manipulador $d=0.70[m]$, además suponemos el caso de una viga en catilibre con donde el par máximo requerido es ejercido en el extremo que sirve de soporte, con lo cual se tiene la configuración mostrada en la figura \ref{Fig:parManip}. Aplicando (ec. 4.2) tenemos un par ejercido por el peso del manipulador de $M = 12.3606[N.m]$.\\

    \begin{equation}
    \begin{aligned}
	\begin{split}
		Par_{carga} & < Par_{suminstrado} \cr
		12.3606 & < 16.8 [N.m]\cr
    \end{split}
    \end{aligned}
    \end{equation}

    Se observó que aún en el caso límite el par suministrado por el arreglo de servomotores en modo paralelo es suficiente para mover el manipulador.\\

    \begin{figure}[H]
		\begin{center}
		\includegraphics[scale=0.50]{manipulator/par_calc.png}
		\end{center}
		\caption{Representación de la descomposición de par para el manipulador de 7DOF en un robot de servicio.}
		\label{Fig:parManip}
	\end{figure}



	Por otro lado es importante conocer la estructura de transmisión de la información de un servomotor con el dispositivo controlador. Para ello los servomotores dynamixel cuentan con un protocolo de comunicación serial asincrono, el cual transmite una serie de datos en una trama, como se ilustra en la figura \ref{manip:trama}.\\

	\begin{figure}[H]
		\begin{center}
		\includegraphics[width=15.5cm, height=1.5cm]{manipulator/trama.png}
		\end{center}
		\caption{Estructura de una trama de información como método de comunicación serial.}
		\label{manip:trama}
	\end{figure}

	El significado de cada byte que compone al paquete datos es el siguiente:\\

	\begin{itemize}
		\itemsep-0.5em
		\item{ \textbf{0xFF 0xFF} :  Es la instrucción que marca el inicio del paquete.}
		\item{ \textbf{ID} :  Es el ID de cada servomotor particular en la cadena.}
		\item{ \textbf{LENGTH} :  Es el tamaño del paquete. El tamaño es calculado como el numero de parámetros más dos (N+2).}
		\item{ \textbf{INSTRUCTION} :  Indica el tipo de instrucción que ejecutará el servomotor dynamixel.}
		\begin{itemize}
			\itemsep-0.5em
			\item{PING}
			\item{READ\_DATA}
			\item{WRITE\_DATA}
			\item{REG\_WRITE}
			\item{ACTION}
			\item{RESET}
			\item{SYN\_WRITE}
		\end{itemize}
		\itemsep-0.5em
		\item{ \textbf{PARANETER 0...N} :  Este parámetro se usa cuando la instrucción requiere datos auxiliares.}
		\item{ \textbf{CHECK SUM} :  Se usa para verificar si el paquete de datos se dañó durante la comunicación.}
	\end{itemize}
	
	A continuación se muestra una tabla con las características más relevantes de los registros que posee cada uno de los servomotores dynamixel.\\

\begin{table}[H]
\centering
\begin{tabular}{ |p{1.2cm}||p{1.8cm}|p{3.0cm}|p{3.8cm}|p{1.8cm}|  }
 \hline
 \textbf{AREA} &\textbf{Dirección} &\textbf{Nombre} &\textbf{Descripción} &\textbf{Acceso}\\
 \hline
   &  3  &	 ID  &   Identificador del motor & $L/E$\\
 \hline
   &  4  &  Baud Rate  & Velocidad transmisión datos dynamixel & $L/E$\\
 \hline
   &  10 &  Drive mode &  Configuraciones en modo dual  & $L/E$ \\
 \hline
   &  14 & Max Torque(L) & Byte bajo del registro Máx torque & $L/E$\\
 \hline
   &  15 & Max Torque(H) & Byte alto del registro Máx torque & $L/E$\\
 \hline
   &  24 & Torque Enable & On \ Off Torque & $L/E$\\
 \hline
   &  26 & D gain & Ganancia de control derivativa & $L/E$\\
 \hline
   &  27 & I gain & Ganancia de control integral   & $L/E$\\
 \hline
   &  28 & P gain & Ganancia de control proporcional& $L/E$\\
 \hline
   &  30 & Goal position(L) & Byte bajo del registro para una posición deseada & $L/E$\\
 \hline
   &  31 & Goal position(H) & Byte alto del registro para una posición deseada   & $L/E$\\
 \hline
   &  32 & Moving Speed(L) & Byte bajo del registro para una velocidad deseada& $L/E$\\
 \hline
   &  33 & Moving Speed(H) & Byte alto del registro para una velocidad deseada& $L/E$\\
 \hline
   &  34 & Torque Limit(L) & Byte bajo del registro para un par deseado& $L/E$\\
 \hline
   &  35 & Torque Limit(H) & Byte alto del registro para un par deseado& $L/E$\\
 \hline
   &  36 & Present position(L) & Byte bajo del registro de la posición presente& L\\
 \hline
   &  37 & Present position(H) & Byte alto del registro de la posición presente& L\\
 \hline
   &  40 & Present load(L) & Byte bajo del registro de la carga presente& L\\
 \hline
   &  41 & Present load(H) & Byte alto del registro de la carga presente& L\\
 \hline
\end{tabular}
\caption{Tabla de registros para servomotores Dynamixel de la gama MX.}
\label{t_manip:5}
\end{table}

		\subsection{Características de conexión}
	Se ha mencionado con anterioridad que el brazo robótico con el cual se realizaron las pruebas correspondientes para este trabajo consta de 10 servomotores dynamixel de la gama MX y de diferentes modelos. Para facilitar la comunicación entre los diferentes modelos de servomotores se utilizó un método de conexión \textit{daysi chain} en cual permite una conexión en cadena entre servomotores, asignándole un ID a cada servomotor y realizando la comunicación a través de un solo puerto.\\

	\begin{figure}[h!]
		\begin{center}
		\includegraphics[width=8.5cm, height=4.5cm]{manipulator/dynamixel_daysi_chain.jpg}
		\caption{Servomotores dynamixel en conexión daisy chain.}
		\end{center}
	\end{figure}

	El sistema \textit{daisy chain} es un esquema de conexiones que forman una sucesión de enlaces, tal que el dispositivo A se encuentra conectado al dispositivo B y este, a su vez, se encuentra conectado a un dispositivo C y así sucesivamente. Es importante señalar que en este tipo de conexión los dispositivos no forman redes, en tal caso el dispositivo C no podría estar conectado al dispositivo A. Otro aspecto importante de resaltar es que en este tipo de conexiones al no poder formar redes la comunicación se realiza dispositivo a dispositivo, por tanto los dispositivos últimos en la cadena pueden presentar retraso o fallas eléctricas con respecto de los dispositivos primeros en orden de la cadena.\\  


%%%%%%%%%%%%%%%%%%%%%%%%%%%%%%%%%%%%%%%%%%%%%%%%%%%%%%%%%%
%%%%%%%%%%%%%%% DESCRIPCION DEL HARDWARE %%%%%%%%%%%%%%%%%
%%%%%%%%%%%%%%%%%%%%%%%%%%%%%%%%%%%%%%%%%%%%%%%%%%%%%%%%%%

	\section{Descripción de los elementos del sistema de manipulación}
		El hardware del manipulador está compuesto por un total de 10 servomotores Dynamixel fabricados por la compañía Robotis \cite{dynamixelEpage}. Esta compañia cuenta con diversas gamas de modelos de servomotores según las necesidades de la aplicación. A continuación se describirá el sistema en orden descendente.\\

		En la parte superior del brazo robótico se encuentran dos servomotores Dynamixel MX-106 conectados como maestro-esclavo, configurados en este modo trabajan de manera conjunta aumentando el par unitario de cada uno de los servomotores. Dadas las características antes descritas en la tabla \ref{t_manip:2}  podemos observar que la configuración de servomotores entrega un par de torsión máximo a rotor bloqueado de $192.1[kg.cm]$ @ 12V. El sentido de giro positivo del arreglo de servomotores es el mostrado en la figura \ref{manip:master}. \\

		La configuración maestro-esclavo es un método de control simultaneo para dos servomotores Dynamixel, esta configuración es sumamente útil cuando se trata de construir una junta cuyo eje de acción es coincidente. Los motores respectivos maestro y esclavo deben estar conectados mediante un cable de sincronización. como se muestra en la figura[4.7]. El servomotor esclavo es directamente controlado por la señal PWM del maestro transmitida a través del cable de sincronización. Es importante mencionar que la información de la posición, velocidad y corriente deseada es ignorada; puesto que esta información depende unicamente del maestro.\\  

		\begin{figure}[H]
			\centering
			\includegraphics[width=11.0cm, height=6.0cm]{manipulator/mx106_MS.png}
			\caption{Configuración maestro-esclavo en servomotores dynamixel.}
			\label{manip:master}
		\end{figure}

		En la configuración maestro-esclavo, el esclavo puede configurarse para adoptar un sentido de giro inverso en caso que el acoplamiento mecánico así lo requiera. Dentro de este modo el servomotor esclavo tendrá la misma velocidad que el motor maestro, solo diferenciado por el sentido inverso de giro.\\

		Posteriormente se encuentra un servomotor modelo MX-106, ubicado en una disposición horizontal. Unido a este servomotor se encuentra un eslabón obtenido mediante técnicas de manufactura aditiva. Dado que la construcción del brazo robótico esta pensada desde el punto de vista antropomórfico,  este conjunto de servomotores, el maestro-esclavo y el servomotor MX-106 describen los grados de libertad que emulan el movimiento de un hombro humano.\\

		\begin{SCfigure}
			\includegraphics[scale=0.35]{manipulator/brazo_4.jpg}
			\caption{Diseño asistido por computadora del brazo robótico del robot Justina. Se realizó en modelado de partes utilizando el software SolidWorks(c). Muestra del ensamble final.}
			\label{Fig:eslabonesDescription}
		\end{SCfigure}


		El eslabón número 1 obtenido mediante técnicas de manufactura aditiva funge como elemento de unión entre servomotores. Al final de este elemento de unión se encuentra un servomotor MX-64 cuyo eje de acción es perpendicular respecto a la base el brazo. Estos tres servomotores describen los movimientos del hombro en comparación con un brazo humano. Directamente acoplado a este servomotor se encuentra otro servomotor modelo MX-106 el cual realiza el símil con un codo humano (figura \ref{Fig:eslabonesDescription}). \\

		Continuando en orden descendente, se encuentra un segundo eslabón utilizado como elemento de unión. En el extremo de este elemento se encuentra un servomotor MX-106, posteriormente un MX-64 y por último un servomotor modelo MX-28. Estos tres elementos construyen un sistema similar al de la muñeca de un brazo humano.\\

	
		Por último, se encuentra el efector final compuesto de dos actuadores MX-28 ubicados en disposición horizontal. En el eje de acción de cada uno de estos se encuentra una pieza obtenida mediante impresión 3D que funge como sujetador para manipular objetos.\\

		Como podemos observar el brazo robótico consta de un total de 10 servomotores conectados en \textit{daisy chain}.
		


%%%%%%%%%%%%%%%%%%%%%%%%%%%%%%%%%%%%%%%%%%%%%%%%%%%%%%%%%%%%%%%%
              %%%%%%  CINEMATICA DIRECTA   %%%%%%%%
%%%%%%%%%%%%%%%%%%%%%%%%%%%%%%%%%%%%%%%%%%%%%%%%%%%%%%%%%%%%%%%%

	\section{Cinemática directa}
El problema de cinemática directa consiste en conocer la posición \textit{(x, y, z)} del efector final dada una determinada configuración de ángulos para el conjunto de actuadores que forman el sistema mecánico. Para ello se describe lo posición del efector final en términos de las transformaciones existentes entre este y la base fija del manipulador.\\


	\subsection{Teoría trasformaciones y medidas del manipulador}
	Se partió del sistema mecánico previamente construido del brazo robótico del robot de servicio Justina. Para ello se realizaron mediciones de distancias entre los respectivos ejes de acción de cada uno de los actuadores a fin describir los desplazamientos de los sistemas de referencia en el brazo robótico.\\

	% #####################################################
	% #####################################################
	% ########   Foto brazo real con medidas\\


	Es importante mencionar que para la descripción de las transformaciones en la plataforma ROS, podemos construir la matriz de rotación de tal manera que incluyamos las rotaciones y desplazamientos en los tres ejes coordenados. Esta característica nos proporciona una ventaja sobre los técnicas de descripción de cinemática directa convencionales, tal es el caso del método Denavit-Hartenberg donde el sistema de transformaciones debe ser descrito utilizando, unicamente, dos rotaciones y dos traslaciones según sus convenciones.\\

	Es importante mencionar esta característica puesto que en algunos casos la convención Denavit-Hartenberg encuentra limitaciones para describir algunas transformaciones arbitrarias. En la representación Denavit-Hartenberg solo hay cuatro parámetros. Pues, mientras que el sistema de referencia i esté rígidamente unido al enlace i, tenemos la libertad para elegir el origen y los ejes de coordenadas del sistema de referencia i+1.\\ 

	Claramente, no es posible representar una transformación homogénea arbitraria usando solo cuatro parámetros. Por lo tanto, comenzamos por determinar qué transformaciones homogéneas se pueden expresar en la forma D\_H.\\

	Supongamos que tenemos dos marcos, indicados por los cuadros 0 y 1, respectivamente. Entonces existe una única matriz de transformación homogénea A que toma las coordenadas del sistema de referencia 1 y las expresa en términos del sistema de referencia 0. Ahora, es necesario aclarar que los sistemas de referencia deben tener dos características adicionales:\\

	\begin{itemize}
		\item{(DH1):  El eje x1 es perpendicular al eje z0}
		\item{(DH2):  El eje x1 intersecta el eje z0} \cite{spong2008robot}\\
	\end{itemize}


    \begin{equation}
	\begin{split}
		T^{n-1}_n = T_{Z_{n-1}}(d_n) * 
			R_z(\theta_n)*T_{x_n} (a_n) * R_{x_n}(\alpha)
    \end{split}
    \end{equation}

    \begin{equation}
	\begin{split}
		T^{n-1}_n = 
		\begin{bmatrix}
\cos \theta_n & -\sin \theta_n \cos \alpha_n & \sin \theta_n \sin \alpha_n &  a_n \cos \theta_n \\
\sin \theta_n & \cos \theta_n \cos \alpha_n & -\cos \theta_n \sin \alpha_n &  a_n \sin \theta_n \\
0     & \sin \alpha_n & \cos \alpha_n &  d_{n} \\
0     &       0        &         0      &    1
		\end{bmatrix}
    \end{split}
    \end{equation}

	Como podemos observar, en la convención de transformaciones Denavit-Hartenberg unicamente requerimos determinar el valor de cuatro variables, dos rotaciones $(\alpha, \theta)$ y dos traslaciones $(a, d)$.\\  


	El sistema de descripción de transformaciones desarrollado por ROS, presenta la ventaja de describir las transformaciones de una manera más detallada al permitirnos describir las rotaciones y traslaciones en los tres ejes. Sin embargo este tipo de soluciones no suele ser las más óptimas, puesto que al incrementarse la cantidad de datos a manejar suele incrementarse el costo computacional que las operaciones demandan. Actualmente con el aumento del poder de procesamiento de las computadoras esta ya no suele ser una limitante.\\

	Para ello se parte de la matriz de rotación compuesta roll, pitch, yaw.\\

	\begin{equation}
	\begin{split}
		R^{0}_1 = R_{z, \phi } * R_{y, \theta } * R_{x, \psi }
	\end{split}
    \end{equation}

    \begin{equation}
	\begin{split}
		R^{0}_1 = 
		\begin{bmatrix}
		\cos\phi  & -\sin\phi &  0 \\
		\sin\phi  &  \cos\phi &  0 \\
		0         &     0     &   1
		\end{bmatrix}*
		\begin{bmatrix}
		\cos\theta   &   0   &  \sin\theta \\
		   0         &   1   &  0 \\
		-\sin\theta  &   0   &  \cos\theta
		\end{bmatrix}*
		\begin{bmatrix}
		   1      &     0    &   0\\
		   0      &   \cos\psi   &  -\sin\psi \\
		   0      &   \sin\psi   &  \cos\psi
		\end{bmatrix}\\
    \end{split}
    \end{equation}

    \begin{equation}
	\begin{split}
		=
		\begin{bmatrix}
		   \cos\phi \cos\theta & -\sin\phi \cos\psi - \cos\phi \sin\theta \sin\psi & \sin\phi \sin\psi + \cos\phi \sin\theta \cos\psi\\
		   \sin\phi \cos\theta & \cos\phi \cos\psi + \sin\phi \sin\theta \sin\psi  & -\cos\phi \sin\psi + \sin\phi \sin\theta \cos\psi \\
		   -\sin\theta         & \cos\theta \sin\psi   &  \cos\theta \cos\psi
		\end{bmatrix}
    \end{split}
    \end{equation}

	\begin{figure}[h!]
		\begin{center}
		\includegraphics[width=6.5cm, height=4.5cm]{manipulator/roll_pitch_yaw.jpeg}
		\caption{Rotaciones roll, pitch, yaw.}
		\end{center}
	\end{figure} 
    

	\subsection{Implementación (Descripción con ROS)}

	Se utilizó un visualizador en 3D para observar el modelo del brazo y verificar el correcto movimiento mecánico del brazo. En tal caso se hizo uso de los paquetes proporcionados por \textit{ROS} para la elaboración de modelos de robots. La descripción del modelo del robot se realizó en un archivo en formato XML. Esta sintaxis permite describir las transformaciones existentes entre cada uno de los sistemas coordenados significativos en el robot, así como permite cargar algún modelo \textit{CAD (Computer Assistand Design)} de las piezas que componen al robot.\\

	\begin{figure}[h!]

		\begin{subfigure}[t]{.5\textwidth}
		\includegraphics[width=5.5cm, height=5.0cm]{manipulator/tf_example.png}
		\end{subfigure}%
		\begin{subfigure}[t]{.5\textwidth}
		\includegraphics[width=5.5cm, height=5.0cm]{manipulator/tf_example2.png}
		\end{subfigure}

		\caption{Ejemplo de un modelo de robot y sus respectivas transformaciones.}
	\end{figure}



	ROS proporciona un paquete para el manejo de información de las transformaciones de un robot que lleva el nombre de \textit{tf}. En un nivel práctico, un árbol de transformación define los desplazamientos en términos de traslación y rotación entre diferentes marcos de coordenadas. La biblioteca \textit{tf} usa una estructura en árbol para garantizar que solo haya un recorrido único que vincule dos sistemas coordinados entre sí y asume que todas las transformaciones del árbol se dirigen desde los nodos primarios a secundarios.\\ 

	Para describir una transformación se requiere definir primero un conector, que funcione como cada uno de los elementos físicos, posteriormente la transformación se encargará de relacionar cada uno de estos conectores. Conceptualmente, cada nodo en el árbol de transformación corresponde a un sistema de coordenadas el cual representa cada uno de los conectores. Por otra parte, cada enlace de conexión corresponde a la transformación que debe aplicarse para pasar del nodo actual a su hijo.\\	

	Para crear el árbol de transformaciones del manipulador de 7DOF, se comenzó por crear un sistema de referencia en la base del brazo, el cual servirá como sistema de referencia fijo, se llamó \textit{base\_ra\_arm}. Posteriormente se fueron agregando eslabones según la configuración real del brazo.\\ 

	\begin{figure}[h!]
		\begin{subfigure}[t]{0.45\textwidth}
		\centering
		\includegraphics[width=6.5cm, height=3.0cm]{manipulator/link_1.png}
		\caption{Constitución básica de \textit{link} en el sistema ROS.}
		\end{subfigure}
		\begin{subfigure}[t]{0.45\textwidth}
		\centering
		\includegraphics[width=3.5cm, height=4.0cm]{manipulator/link_description.png}
		\caption{Ejemplo de sistema de transformaciones (sistemas de referencia y links).}
		\end{subfigure}
	\end{figure} 

	Para crear esta configuración entre los diferentes sistemas de referencia, primero debemos decidir qué nodo será el padre y cuál será el hijo. Esta distinción es importante porque asume que todas las transformaciones se mueven de padres a hijos. Elegimos el sistema de referencia \textit{base\_ra\_arm} como principal, ya que a medida que se agregan otras piezas: motores, eslabones y conectores, tendrá más sentido que se relacionen con un sistema fijo. Esto significa que la transformación asociada con el efector final y el sistema \textit{base\_ra\_arm} puede ser obtenida mediante el árbol de transformaciones configurado. La conversión de la posición del sistema de referencia ubicado en el efector final al sistema de referencia fijo ubicado en la base del brazo puede ser obtenida mediante una llamada a la biblioteca tf.\\

	\subsection{Comparación entre la convención D-H y la descripción solo con rotaciones y traslaciones.}

	Se implementaron dos metodologías de descripciones para el cálculo de la cinemática directa del brazo robótico. La primera de ellas consistió en realizar la descripción completa de las transformaciones utilizando la información de rotación y traslación de cada uno de los respectivos sistemas de referencia.\\

	En este proceso y con la información previamente obtenida de las dimensiones del brazo robótico se construyó la siguiente tabla donde se muestra la información de los desplazamientos y rotaciones de los respectivos sistemas de referencia.\\

\begin{table}[H]
\centering
\begin{tabular}{ |p{1.0cm}|p{1.5cm}|p{1.5cm}|p{1.5cm}|p{1.5cm}|p{1.5cm}|p{1.5cm}|}
 \hline
  & \multicolumn{3}{|c|}{\textbf{TRANSLACIÓN}} & \multicolumn{3}{|c|}{\textbf{ROTACIÓN}} \\
 \hline
  &  x & y & z & roll & pitch & yaw\\
 \hline
 1 &  0.065 & 0.00 & 0.000 & $\pi /2$  &   0.000   & 0.000    \\
 \hline
 2  & 0.215 & 0.00 & 0.000 &   0.000   & $\pi /2$  & 0.000    \\
 \hline
 3  & 0.000 & 0.00 & 0.060 & $-\pi /2$ & $-\pi /2$ & 0.000    \\
 \hline
 4  & 0.190 & 0.00 & 0.000 & $\pi /2$  &   0.000   & $\pi /2$ \\
 \hline
 5  & 0.000 & 0.00 & 0.036 & $-\pi /2$ & $-\pi /2$ &  0.000   \\
 \hline
 6  & 0.095 & 0.00 & 0.000 & $\pi /2$  &   0.000   & $\pi /2$ \\
 \hline
\end{tabular}
\caption{Tabla de parámetros de transformaciones entre sistemas de referencia para el brazo robótico de 7DOF.}
\label{t_manip:6}
\end{table}

	Con la información obtenida de las dimensiones del brazo robótico real se pudo realizar un modelo virtual del brazo utilizando un archivo en formato XML donde se describen las transformaciones entre los diferentes sistemas de referencia y los elemetos de conexión obtenidos de modelos CAD. El resultado se puede observar en la figura \ref{Fig:descTrans}.\\ 

	El formato de descripción virtual de un modelo de robot en formato XML se compone de \textit{links} y \textit{joints}. Un link describe las características de un elemento de unión, estas pueden ser mecánicas, de material, de posición, etc. Un \textit{joint} por otra parte describe la relación existente entre los orígenes de dos links, dicho de otro modo describe la transformación existente entre los sistemas de referencias (\textit{orígenes}) de dos links.\\

	\vspace{0.7in}

	\begin{figure}[H]
		\begin{subfigure}[t]{.5\textwidth}
		\centering
		\includegraphics[width=6.0cm, height=10.5cm]{manipulator/brazo_lateralView.png}
		\caption{Vista lateral.}
		\end{subfigure}%
		\begin{subfigure}[t]{.5\textwidth}
		\centering
		\includegraphics[width=3.2cm, height=10.5cm]{manipulator/brazo_posterior.png}
		\caption{Vista trimetrica.}
		\end{subfigure}
		\caption{Modelo virtual del brazo robótico.}
		\label{Fig:descTrans}
	\end{figure}

	\newpage
	\begin{center}
	\begin{lstlisting}[language=html, caption=Ejemplo de un archivo URDF para la descripción del sistema de manipulación.]
<link name="base_ra_arm">
    <visual>
      <origin xyz="-0.01 0.0 0.0" rpy="1.5707 0.0 1.5707"/>
      <geometry> 
        <mesh filename="package://knowledge/hardware/stl/brazo/mx106_2.stl"/>
      </geometry>
      <material name="black_gray"><color rgba="0.2 0.2 0.2 1"/></material>
    </visual>
  </link>

  <link name="ra_link0">
    <visual>
      <origin xyz="0.0 0.0 0.005" rpy="0.0 -1.5707 0.0"/>
      <geometry> 
        <mesh filename="package://knowledge/hardware/stl/brazo/mx106_s.stl"/>
      </geometry>
      <material name="gray_black"><color rgba="0.3 0.3 0.3 1"/></material>
    </visual>
  </link>

  <link name="ra_link1">
    <visual>
      <origin xyz="0.0 0.0 0.0" rpy="1.5707 -1.5707 0.0"/>
      <geometry> 
        <mesh filename="package://knowledge/hardware/stl/brazo/bone_1.stl"/>
      </geometry>
      <material name="ra_material"><color rgba="0.9 0.85 0.75 1"/></material>
    </visual>

  </link>
  .
  .
  .
   <joint name="ra_1_joint" type="revolute">
    <origin xyz="0.0 0.0 0.0" rpy="0.0 0.0 0.0"/>
    <parent link="base_ra_arm"/>
    <child link="ra_link0"/>
    <limit effort="0.0" lower="0.0" upper="0" velocity="0.0"/>
    <axis xyz="0 0 1"/>
  </joint>

  <joint name="ra_2_joint" type="revolute">
    <origin xyz="0.064 0.0 0.0" rpy="1.5707 0.0 0.0"/>
    <parent link="ra_link0"/>
    <child link="ra_link1"/>
    <limit effort="0.0" lower="0.0" upper="0" velocity="0.0"/>
    <axis xyz="0 0 1"/>
  </joint>
  .
  .
  .
\end{lstlisting}
\end{center}


		

	Por otro lado se realizó la descripción del sistema de manipulación utilizando la convención Denavit-Hartenberg. Para ello se comenzó por plantear los ejes $z$ que deben ser coincidentes, en posición y en sentido de giro, con los ejes de acción de cada uno de los actuadores. Sin embargo, en este punto se observó una limitante en el método D-H, al intentar describir la trasformación entre el $link_2$ y el $link_3$ se observó que el método no puede describir con suficiente precisión la rotación en el eje $"x"$ y al mismo tiempo un desplazamiento en el $eje$ $y$. La manera de solucionar esta limitante consiste en realizar una transformación de rotación en el eje x primero, con esto los dos frames compartirán el mismo origen del sistema de referencia, posteriormente se plantea la traslación, ahora en el $eje$ $z$.\\
	
	\begin{figure}[h!]
		\begin{center}
		\includegraphics[width=6.5cm, height=6.0cm]{manipulator/DH_transform_ROS.png}
		\caption{Ejemplo de  obtención de parametros DH.}
		\end{center}
	\end{figure} 

	Por tanto, los parámetros que describen las transformaciones del brazo según la convención D-H, se muestra en la siguiente tabla:\\

\begin{table}[H]
\centering
\begin{tabular}{ |p{1.5cm}|p{1.5cm}|p{1.5cm}|p{1.5cm}|p{1.5cm}|}
 \hline
 \multicolumn{5}{|c|}{\textbf{PARAMETROS DENAVIT-HARTENBERG}} \\
 \hline
      &    D  &   A   & $\alpha$  &  $\theta$ \\
 \hline
 0-1  & 0.000 & 0.065 & $\pi /2$  &   0.000   \\
 \hline
 1-2  & 0.000 & 0.000 & $\pi /2$  & $\pi /2$  \\
 \hline
 2-3  & 0.275 & 0.000 & $-\pi /2$ & $-\pi /2$ \\
 \hline
 3-4  & 0.000 & 0.000 & $\pi /2$  &   0.000   \\
 \hline
 4-5  & 0.226 & 0.000 & $-\pi /2$ &   0.000   \\
 \hline
 5-6  & 0.000 & 0.000 & $\pi /2$  &   0.000   \\
 \hline
 6-7  & 0.165 & 0.000 &   0.000   &   0.000   \\
 \hline
\end{tabular}
\caption{Tabla de parámetros D-H para brazo robótico antropomórfico de 7DOF.}
\label{t_manip:7}
\end{table}

\begin{figure}[H]
	\begin{subfigure}[t]{.5\textwidth}
	\centering
	\includegraphics[width=2.5cm, height=8.5cm]{manipulator/DH_1.png}
	\caption{Vista trimétrica.}
	\end{subfigure}%
	\begin{subfigure}[t]{.5\textwidth}
	\centering
	\includegraphics[width=5.50cm, height=9.5cm]{manipulator/DH_links.png}
	\caption{Vista de transformaciones.}
	\end{subfigure}
	\caption{Modelo virtual del brazo robótico utilizando la convención DH.}
	\label{Fig:transDH}
\end{figure}


	Como se puede observar en la Figura \ref{Fig:transDH}, la descripción de las transformaciones entre el $link 1$ y el $link 2$ comparten el origen. Unicamente difieren en la rotación del respectivo sistema de referencia. Sucede lo mismo para describir la transformación entre las parejas de $links$:  $[3 - 4]$ y $[5 - 6]$.\\

	De la comparación de estas dos representaciones podemos obtener pros y contras de cada una de ellas. Por un lado una representación completa de las trasformaciones nos da la posibilidad de realizar una descripción más apegada al movimiento real del robot. Esta característica permite alinear cada uno de los sistemas de referencia a el eje de movimiento de cada uno de los actuadores. En este sentido, es conveniente utilizar esta descripción puesto que simplifica la manera de publicar cada una de las transformaciones entre los sistemas de referencia, basta con estar leyendo la posición de cada uno de los actuadores y publicar dicho valor para obtener la representación gráfica de la configuración del brazo en el modelo virtual.\\

	Por otro lado, estar escuchando en tiempo real siete transformaciones completas requiere un mayor costo computacional comparado con realizar la multiplicación de las mismas siete matrices de transformación solo cuando sea de interés conocer la posición del efector final. En tal caso resulta conveniente tener las dos descripciones: la descripción completa para el despliegue de información de manera visual en el ambiente virtual y la descripción D-H para conocer la posición del efector final cuando sea necesario.\\




%%%%%%%%%%%%%%%%%%%%%%%%%%%%%%%%%%%%%%%%%%%%%%%%%%%%%%%%%%%%%%%%
  %%%%%%%%%%%%%    DETERMINACIÓN DEL WORKSPACE    %%%%%%%%%%%%
%%%%%%%%%%%%%%%%%%%%%%%%%%%%%%%%%%%%%%%%%%%%%%%%%%%%%%%%%%%%%%%%
	\section{Determinación del área de trabajo}

	El espacio de trabajo es otro parámetro importante el cual servirá para caracterizar las dimensiones del brazo robótico con el cual estamos trabajando. El análisis del espacio de trabajo de robots manipuladores es de gran interés, puesto que la geometría del espacio de trabajo puede considerarse no sólo un aspecto fundamental para el diseño del robot sino que también es esencial para la ubicación del robot en el entorno de trabajo y también para la planificación de trayectorias.\\

	En robótica, el término espacio de trabajo puede ser entendido como: \textit{El grupo de puntos que pueden ser alcanzados por su efector-final} \cite{cao2011accurate}.\\

	Dicho de otro modo, el espacio de trabajo de un robot es el espacio en el cual el mecanismo puede trabajar según sus propias restricciones mecánicas.\\

\begin{figure}[h!]
	\centering
	\begin{subfigure}[t]{.55\textwidth}
	\centering
	\includegraphics[width=4.5cm, height=4.5cm]{workSpace/ws_Yi_Cao_1.png}
	\caption{Nube de puntos del espacio de trabajo en 2D.}
	\end{subfigure}%
	\begin{subfigure}[t]{.55\textwidth}
	\centering
	\includegraphics[width=5.00cm, height=4.3cm]{workSpace/ws_Yi_Cao_2.png}
	\caption{Curva de la frontera del espacio de\\ trabajo del brazo robótico.}
	\end{subfigure}
	\caption{Espacio de trabajo de un robot 3R, donde los puntos  de color azul representan el espacio de trabajo de este robot. Como se ha dicho previamente, los puntos en color azul representan cada una de las posiciones que puede alcanzar el efector-final del robot 3R.}
\end{figure}


\textbf{Principales características de un espacio de trabajo.}\\

	Cuando se pretende estudiar un espacio de trabajo, lo más importante es su forma y volumen (dimensiones y estructura). Ambos aspectos tienen una importancia significativa debido al impacto que éstos ejercen en el diseño del robot y también en su manipulabilidad. En el presente trabajo se dará una mayor importancia al estudio de las dimensiones del espacio de trabajo desde el punto de vista de la manipulabilidad.\\

	En el caso del presente trabajo fue preciso conocer las características de  forma, dimensiones y estructura del robot a analizar, puesto que de estas características dependerá su espacio de trabajo.\\

	En el apartado 4.6 se aborda el cálculo de la cinemática inversa por dos metodologías diferentes, se observó que para el caso del algoritmo numérico con la paquetería MoveIt! el algoritmo tiene un tiempo de ejecución muy alto cuando la posición deseada está fuera del espacio de trabajo del manipulador. Por ello, es importante caracterizar dicho espacio de trabajo. Es difícil validar si un punto se encuentra dentro del espacio semi-esférico (Figura \ref{Fig:wsJustina}); se planteó una región prismática para facilitar esta validación.\\ 

\begin{itemize}
	\itemsep-0.5em
	\item{La forma es importante para la definición del entorno donde el robot trabajará.}
	\item{Las dimensiones son importantes para la determinación del alcance del efector-final.}
	\item{La estructura del espacio de trabajo es importante para asegurar las características cinemáticas del robot las cuales están relacionadas con la interacción entre el robot y el entorno.}
\end{itemize}

	Además, la forma, dimensiones y estructura del espacio de trabajo dependen de las propiedades del robot en cuestión. Las dimensiones de los eslabones del robot y las limitaciones mecánicas de las articulaciones tienen una gran influencia en las dimensiones del espacio de trabajo. La forma depende de la estructura geométrica del robot y también de las propiedades de los grados de libertad. Por otro lado la estructura del espacio de trabajo viene definida por la estructura del robot y las dimensiones de sus eslabones.\\


%%%    ########################################   %%%%%%%%
%%% IMPLEMENTACION: OBTENCIÓN DEL WS DEL BRAZO ROBOTICO
%%%    ########################################   %%%%%%%%

	Para la obtención del espacio de trabajo del brazo robótico en cuestión fue preciso partir de las características antes mencionadas. En este aspecto conocemos las tres características necesarias para la obtención del espacio de trabajo del robot, en cuanto a la forma podemos mencionar que el robot en cuestión es antropomórfico de 7 grados de libertad, cuyas dimensiones las hemos obtenido con anterioridad. En cuanto a la estructura podemos mencionar que conocemos el modelo de la cinemática directa del brazo e incluso que podemos visualizar en tiempo real la configuración del brazo antropomórfico, y por ende conocer la posición del efector final. Solo hace falta conocer las restricciones mecánicas del brazo robótico que se muestran en la tabla \ref{t_manip:8} .\\

\begin{table}[H]
\centering
\begin{tabular}{ |p{0.5cm}|p{1.5cm}|p{1.5cm}|}
 \hline
 \multicolumn{3}{|c|}{\textbf{RESTRICCIONES MECÁNICAS }} \\
 \multicolumn{3}{|c|}{\textbf{DEL BRAZO ROBÓTICO [RAD]}} \\
 \hline
    &  $\theta_1 $ & $[1.47, -1.47]$ \\
 \hline
    &  $\theta_2 $ & $[1.25, -0.21]$ \\
 \hline
    &  $\theta_3 $ & $[1.5707, -1.5707]$ \\
 \hline
    &  $\theta_4 $ & $[2.15, -1.15]$ \\
 \hline
    &  $\theta_5 $ & $[1.5707, -1.5707]$ \\
 \hline
    &  $\theta_6 $ & $[1.5, -1.35]$ \\
 \hline
    &  $\theta_7 $ & $[1.5707, -15707]$ \\
 \hline
\end{tabular}
\caption{Tabla de restricciones mecánicas para brazo robótico antropomórfico de 7DOF.}
\label{t_manip:8}
\end{table}

	Dentro de las metodologías planteadas para la obtención del espacio de trabajo de un robot existen diversas vertientes, una de ellas menciona la posibilidad de obtener dicho espacio generando números aleatorios con una distribución normal acotando los valores aleatorios dentro de las restricciones mecánicas. Posteriormente, con los valores aleatorios de ángulos y el modelo de la cinemática directa del brazo robótico, se obtiene la posición del efector final y se gráfica cada uno de estos valores en el tiempo.\\ %%\textbf{****cite}.\\


\begin{figure}[H]
	\centering
	\includegraphics[width=11.00cm, height=7.0cm]{workSpace/esquema_WS2.png}
	\caption{Esquema de construcción del espacio de trabajo para un robot de 7DOF.}
\end{figure}

	Sin embargo, para el desarrollo de este documento se trabajó tanto con el modelo real del brazo robótico en cuestión, así como con el modelo virtual del mismo. Dadas estas condiciones se planteó el desarrollo de la siguiente manera: con el modelo real del brazo se obtuvieron en tiempo real las lecturas de posición de cada uno de los actuadores, posteriormente se actualizó el modelo virtual del mismo. Con esta información previa, los valores de posición de cada uno de los actuadores y el modelo de la cinemática directa del brazo, se calculaba en tiempo real la posición del efector final.\\


\begin{figure}[H]
	\centering
	\begin{subfigure}[t]{.55\textwidth}
	\centering
	\includegraphics[width=4.50cm, height=4.5cm]{workSpace/ws_9.png}
	\caption{Vista superior.}
	\end{subfigure}%
	\begin{subfigure}[t]{.55\textwidth}
	\centering
	\includegraphics[width=4.20cm, height=4.5cm]{workSpace/ws_8.png}
	\caption{Vista lateral.}
	\end{subfigure}
	\caption{Nube de puntos del espacio de trabajo del robot de servicio Justina. Nube de puntos desplegada en el visualizador gráfico Rviz, incluido en la paquetería de ROS.}
	\label{Fig:wsJustina}
\end{figure}
	
	La técnica consiste en mover el manipulador en tiempo real, calcular la cinemática directa y obtener la posición del efector final para cada instante.\\

	La ventaja que esta técnica supone, sobre otras, es la capacidad de obtener el espacio de trabajo tomando en cuenta las restricciones reales del robot, así como observar las trayectorias y los puntos críticos del efector final. En este sentido es importante mencionar que para algunos casos el efector final podría alcanzar un punto dentro del área de trabajo; sin embargo esta podría suponer un esfuerzo indeseado en alguno de los actuadores. Con esta técnica se pudo evitar ese tipo de configuraciones, y construir el espacio de trabajo bajo estos supuestos.\\

	Se obtuvo la nube de puntos que en forma se puede aproximar a la sección de una esfera, como se observa en la figura \ref{Fig:wsJustina} . En el desarrollo de este trabajo se planteó la problemática de conocer si el brazo robótico es capaz de alcanzar un punto especifico en el espacio, en tal caso los desarrollos en este punto deberían abordar la temática de la siguiente manera: encontrar la ecuación de un volumen que se aproxime a la forma que posee la nube de puntos con el objetivo de saber con certeza si el punto pertenece al volumen de trabajo del manipulador.\\

	El problema de conocer aquellos puntos alcanzables para el brazo robótico se abordó de la siguiente manera; se propuso una región en el espacio con forma prismática. La región en el espacio mencionada con anterioridad cumple la condición que sus vértices pertenecen a la nube de puntos del espacio de trabajo del robot, con esto se asegura que todos los puntos dentro de la región propuesta son alcanzables para el robot.\\ 

	Sin perder de vista el objetivo final de este trabajo que es la correcta manipulación de objetos con cierta rotación en ``z", este objetivo añade una limitante más: ``La región en el espacio propuesta debe contener puntos donde el efector final pueda llegar con una orientación en $z$  de por lo menos $\pi/2$ respecto sistema de referencia del robot". Para ello restamos a el valor máximo en el eje ``x" del espacio de trabajo la cantidad $0.165[m]$ según los parámetros DH. De esta manera se consiguió obtener un espacio de trabajo acotado cuyas características se observan en la figura \ref{Fig:wsJustina_prism} y en la tabla \ref{t_manip:9}.\\

\begin{figure}[H]
	\centering
	\begin{subfigure}[t]{.55\textwidth}
	\centering
	\includegraphics[width=4.50cm, height=5.5cm]{workSpace/ws_bb11.png}
	\caption{Vista superior.}
	\end{subfigure}%
	\begin{subfigure}[t]{.55\textwidth}
	\centering
	\includegraphics[width=4.50cm, height=5.5cm]{workSpace/ws_bb16.png}
	\caption{Vista lateral.}
	\end{subfigure}
	\caption{Espacio de trabajo acotado de un robot antropomórfico de 7DOF. Puntos en el espacio que aproximan a un prisma el espacio de trabajo.}
	\label{Fig:wsJustina_prism}
\end{figure}

	En la siguiente tabla podemos observar los parámetros que caracterizan el espacio de trabajo acotado para el brazo robótico de 7DOF.\\

\begin{table}[H]
\centering
\begin{tabular}{ |p{0.5cm}|p{1.5cm}|p{1.5cm}|}
 \hline
 \multicolumn{3}{|c|}{\textbf{PARAMÉTROS DEL ESPACIO DE TRABAJO}} \\
 \multicolumn{3}{|c|}{\textbf{DEL BRAZO ROBÓTICO}} \\
 \hline
 \multicolumn{3}{|c|}{\textbf{VERTICES} p(x, y, z) Respecto del robot} \\
 \hline
    &  $P_1 $ & $[0.165837,   0.064973,  0.719878]$ \\
 \hline
    &  $P_2 $ & $[0.165837,  -0.57737,   0.719878]$ \\
 \hline
    &  $P_3 $ & $[0.426291,  -0.57737,   0.719878]$ \\
 \hline
    &  $P_4 $ & $[0.426291,   0.064973,  0.719878]$ \\
 \hline
    &  $P_5 $ & $[0.606291,   0.064973,  1.09449]$ \\
 \hline
    &  $P_6 $ & $[0.165837,   0.064973,  1.09449]$ \\
 \hline
    &  $P_7 $ & $[0.165837,  -0.57737,   1.09449]$ \\
 \hline
    &  $P_8 $ & $[0.165837,  -0.57737,   1.09449]$ \\
 \hline
 \multicolumn{3}{|c|}{\textbf{VOLUMEN}} \\
 \hline
 	&  $V[m^3]$ &   0.0838517      \\
 \hline
\end{tabular}
\caption{Tabla de características del espacio de trabajo para brazo robótico antropomórfico de 7DOF. Puntos en el espacio que aproximan a un prisma el espacio de trabajo.}
\label{t_manip:9}
\end{table}	





%%%    ########################################   %%%%%%%%
%%% OBTENCIÓN DEL WS DEL BRAZO ROBOTICO DEL ALGORITMO
%%%    ########################################   %%%%%%%%


%\newpage
	\section{Cinemática inversa}
	El problema de la cinemática inversa consiste en, dada una posición deseada $p(x, y, z, roll, pitch, yaw)$, encontrar el valor de cada una de las juntas cinemáticas que lleven al brazo a la posición deseada. Este problema suele ser uno de los más interesantes y complejos, dentro del estudio de la robótica. Es preciso mencionar que para llevar el efector final a cualquier posición y orientación deseada se requiere por lo menos un brazo robótico de 6DOF. En el desarrollo de este trabajo se plantea la resolución de la cinemática para un manipulador de 7DOF, lo cual suele aumentar el grado de complejidad en la resolución del problema. Dadas una posición y orientación deseadas expresadas en la matriz:\\

\begin{equation}
\begin{split}
	H = 
	\begin{bmatrix}
	R  &  0  \\
	0  &  1  \\ 
	\end{bmatrix}
\end{split}
\end{equation}

	El problema de la cinemática inversa consiste en encontrar una o todas las soluciones a la ecuación:\\

\begin{equation}
	T^{0}_{n}(q_1, q_2...  q_n) = H
\end{equation}	
	
	donde $q_1... q_n$ son los valores de las respectivas juntas mecánicas. Representado de otra manera:\\ 

\begin{equation}
	f(q_1, q_2...  q_n) = g(x, y, z, roll, pitch, yaw) 
\end{equation}	


	Para abordar la resolución de dicho problema se suele hacer uso del concepto de \textit{desacople cinemático}. Utilizando este concepto se puede simplificar el problema del cálculo completo de la cinemática inversa. Por medio del desacople cinemático se pueden considerar de manera independiente el cálculo de la orientación inversa y el cálculo de la posición inversa.\\

	\subsection{Desacople cinemático}
	Como se ha mencionado con anterioridad, el problema de la cinemática inversa puede resolverse por separado en la resolución de la orientación inversa y la posición inversa, para manipuladores de 6 DOF o más. Es importante mencionar que para utilizar esta metodología de resolución el manipulador debe poseer la configuración de \textit{muñeca esférica}. Es necesario partir de esta premisa puesto que la configuración de muñeca esférica implica que los tres últimos ejes de acción de intersectan en un punto llamado $O_c$. Además estos tres últimos ejes de acción pueden determinar la orientación del efector final sin afectar la posición alcanzada por los ángulos anteriores.\\

\begin{equation}
	R^0_7(q_1,.....  ,q_7) = R 
\end{equation}

\begin{equation}
	o^0_7(q_1,.....  ,q_7) = o 
\end{equation}

	La suposición de una \textit{muñeca esférica} implica que los ejes $z_4$, $z_5$ y $z_6$ se intersectan en $o_c$. Como se observa en la figura \ref{Fig:wristCenter} el origen del sistema $O_5$ coincide con el centro de la muñeca esférica $O_c$. Es importante resaltar que el movimiento de los tres últimos ejes no afecta la posición del centro de la muñeca $O_c$. 

\begin{figure}[H]
	\centering
	\begin{subfigure}[t]{.55\textwidth}
	\centering
	\includegraphics[width=2.50cm, height=7.0cm]{manipulator/wrist_center.png}
	\caption{Vista frontal.}
	\end{subfigure}%
	\begin{subfigure}[t]{.55\textwidth}
	\centering
	\includegraphics[width=5.00cm, height=7.0cm]{manipulator/wrist_center2.png}
	\caption{Vista lateral.}
	\end{subfigure}
	\caption{Centro de rotación de una muñeca esférica.}
\end{figure}

	El centro del efector final deseado podemos llamarlo \textit{o} y se puede obtener como la traslación $d_6$ a lo largo del eje \textit{x} a partir del punto $o^0_c$, donde $d_6$ es la distancia del centro del efector final al punto donde se intersectan los ejes de acción $z_4$, $z_5$ y $z_6$.\\

\begin{equation}
o = o^0_c + d_6 R \hat{x} 
\end{equation}

\begin{equation}
o^0_c = o - d_6 R \hat{x} 
\end{equation}

\begin{equation}
\begin{split}
	\begin{bmatrix}
	x_{oc}\\
	y_{oc}\\
	z_{oc} 
	\end{bmatrix}
	=
	\begin{bmatrix}
	x_{EE} - d_6 r_{13}\\
	y_{EE} - d_6 r_{23}\\
	z_{EE} - d_6 r_{33}
	\end{bmatrix}
\end{split} 
\end{equation}

	De la ecuación 4.13 podemos observar que para realizar el cálculo del punto que representa el centro de la muñeca del manipulador solo debemos conocer el punto objetivo del efector final $P_{EE}$ y la orientación deseada \textit{R}. Utilizando la ecuación 4.13 calculamos el centro de la muñeca \textit{(punto en rojo)}. Posteriormente este punto servirá para calcular la cinemática inversa del manipulador quedando unicamente 4 valores de articulaciones por calcular.\\

\begin{figure}[H]
	\centering
	\begin{subfigure}[t]{.55\textwidth}
	\centering
	\includegraphics[width=3.50cm, height=7.0cm]{manipulator/wristCenter_yaw.png}
	\caption{Posición deseada del efector final. (verde)}
	\end{subfigure}%
	\begin{subfigure}[t]{.55\textwidth}
	\centering
	\includegraphics[width=3.00cm, height=7.0cm]{manipulator/wristCenter_yaw2.png}
	\caption{Posición del centro de la muñeca. (rojo)}
	\end{subfigure}
	\caption{Centro de la muñeca con rotación $\pi/2$.}
	\label{Fig:wristCenter}
\end{figure}

	Para motivos de este trabajo se utilizó un algoritmo ya implementado con anterioridad en el robot de servicio Justina, el cual utiliza un método geométrico para la resolución de esta tarea. El propósito de este trabajo es la comparación entre algoritmos para resolver cinemáticas inversas por tanto no se abordará a fondo el desarrollo del algoritmo; sin embargo a continuación se enlistan las ecuaciones y consideraciones utilizadas en este algoritmo: \\

\begin{equation}
\begin{aligned}
\begin{split}
	r & =  \sqrt{x^2 + y^2 + (z-D_1)^2 }\cr
\end{split} 
\end{aligned}
\end{equation}

\begin{equation}
\begin{aligned}
\begin{split}
	\alpha & =  Atan2( (z- D_1), \sqrt{x^2 + y^2})\cr
	\gamma & = \arccos{ \frac{-D^2_2 - D^2_3 + r^2}{-2D_2 \; D_3} } \cr
	\beta  & = \frac{ \arcsin{D_3} \sin{\gamma} } {r} \cr
\end{split} 
\end{aligned}
\end{equation}

Esta solución siempre considera solo la solución de codo arriba:\\

\begin{equation}
\begin{aligned}
\begin{split}
		tunningRadiusElbow = D_2 \; \sin{\beta}  \cr
\end{split} 
\end{aligned}
\end{equation}

Se utiliza un vector que representa la posición del codo con respecto al sistema del mismo codo:\\ 

\begin{equation}
\begin{aligned}
\begin{split}	
	p_{elbow}[0] & = 0 \cr
	p_{elbow}[1] & = -tunningRadiusElbow * \cos{(elbowAngle)} \cr
	& = D_2*\sin{(\beta)}*\cos{(elbowAngle)} \cr
	p_{elbow}[2] & = -tunningRadiusElbow * \sin{(elbowAngle)} \cr
	& = D_2*\sin{(\beta)}*\sin{(elbowAngle)} \cr
\end{split} 
\end{aligned}
\end{equation}	

Posteriormente se define una matriz de transformación del sistema sobre el cual gira el codo al sistema base del brazo:\\

\begin{equation}
\begin{aligned}
\begin{split}	
	oReRot =
	\begin{bmatrix}
	\cos{( articular[0] )}*\cos{\alpha}  &  -\sin{( articular[0] )} & \cos{( articular[0] )} * \sin{-\alpha} \\
	\sin{( articular[0] )}*\cos{\alpha}  &  \cos{( articular[0]) } &  \sin{( articular[0] )} * \sin{-\alpha}\\
	-\sin{\alpha}  &  0 &  \cos{-\alpha}\\ 
	\end{bmatrix}
\end{split} 
\end{aligned}
\end{equation}

\begin{equation}
\begin{aligned}
\begin{split}	
	oReTrans =
	\begin{bmatrix}
	dhD[2]*\cos{\beta}*\cos{\alpha}*\cos{articular[0]}\\
	dhD[2]*\cos{\beta}*\cos{\alpha}*\sin{articular[0]}\\
	dhD[2]*\cos{\beta}*\sin{\alpha}+dhD[0]\\ 
	\end{bmatrix}
\end{split} 
\end{aligned}
\end{equation}

\begin{equation}
\begin{aligned}
\begin{split}	
	oRe =
	\begin{bmatrix}
	oReRot & oReTrans\\
	0 & 1\\ 
	\end{bmatrix}
\end{split} 
\end{aligned}
\end{equation}	

Se transforma la posición del codo respecto al sistema base:\\
\begin{equation}
\begin{aligned}
\begin{split}	
	p_{elbow} = oRe * p_{elbow}\\
\end{split} 
\end{aligned}
\end{equation}

\begin{equation}
\begin{aligned}
\begin{split}	
	articular[0] & = atan2(\frac{p_{elbow}[1]+dhA[0]*\sin{articular[0]} }{p_{elbow}[0]+dhA[0]*\cos{articular[0]}})\cr 
	articular[1] & = atan2(\frac{p_{elbow}[2]-dhD[0]} {\sqrt{p_{elbow}[0]^2 + p_{elbow}[0]^2}})\cr 
	articular[2] & = 0 \cr 
	articular[3] & = 0 \cr 
\end{split} 
\end{aligned}
\end{equation}

Con estos valores se obtiene la transformación del $link_0$ al $link_4$ para, posteriormente, calcular los dos ángulos restantes, $articular[2]$ y $articular[3]$:

\begin{equation}
\begin{aligned}
\begin{split}	
	T_{0}^{4} = T_{0}^{1} * T_{1}^{2} * T_{2}^{3} * T_{3}^{4} \cr
\end{split} 
\end{aligned}
\end{equation}

donde, cada una de las transformaciones se encuentra definida como:\\

\begin{equation}
\begin{aligned}
\begin{split}	
	T_{i}^{i+1} \longleftarrow & \{ \; Rot=RPY(dhAlpha[i],\; 0,\; articular[i]+dhTheta[i] ), \cr 
	& Trans=(dhA[i]*\cos{articular[i]},\; dhA[i]*\sin{articular[i]},\; dhD[i] ) \; \} 
\end{split} 
\end{aligned}
\end{equation}

posteriormente se obtiene la matriz de transformación inversa, para calcular la posición de la muñeca:\\ 

\begin{equation}
\begin{aligned}
\begin{split}	
	T_{4}^{0} = ( T_{0}^{4} ) ^ {-1}\cr \cr
	wristPos = T_{4}^{0} * wristPos\cr
\end{split} 
\end{aligned}
\end{equation}

por tanto se tiene:\\

\begin{equation}
\begin{aligned}
\begin{split}	
	articular[2] & = atan2( \frac{wristPos[1]}{wristPos[2]} ) \cr \cr
	articular[3] & = \pi/2 \; - atan2(\frac{wristPos[2]} { \sqrt{wristPos[0]^2 + wristPos[1]^2 } } )\cr
\end{split} 
\end{aligned}
\end{equation}



	Como salida del algoritmo se tiene:\\

\begin{equation}
\begin{aligned}
\begin{split}	
	articular[0] & = atan2(\frac{p_{elbow}[1]+dhA[0]*\sin{articular[0]} }{p_{elbow}[0]+dhA[0]*\cos{articular[0]}})\cr 
	articular[1] & = atan2(\frac{p_{elbow}[2]-dhD[0]} {\sqrt{p_{elbow}[0]^2 + p_{elbow}[0]^2}} \cr 
	articular[2] & = atan2( \frac{wristPos[1]}{wristPos[2]} ) \cr 
	articular[3] & = \pi/2 \; - atan2(\frac{wristPos[2]} { \sqrt{wristPos[0]^2 + wristPos[1]^2 } } )
\end{split} 
\end{aligned}
\end{equation}	

posteriormente, se calculan los ángulos deseados de la muñeca.\\


\IncMargin{1em}
		\begin{algorithm}[H]
		\SetKwData{Left}{left}\SetKwData{This}{this}\SetKwData{Up}{up}
		\SetKwFunction{Union}{Union}\SetKwFunction{FindCompress}{FindCompress}
		\SetKwInOut{Input}{input}\SetKwInOut{Output}{output}
		
		\BlankLine
		\BlankLine
		\BlankLine

		 \If{$1 - fabs(R^4_7[0,0]) \; < \; 0.0001$}
		 {
		 	$articular[4] = 0$ \;
			$articular[5] = 0$ \;
			$articular[6] = Atan2( R^4_7[1,1] ,\;  R^4_7[0,1]$ )\;
		 }
		\Else
		{
			$articular[4] = Atan2(R^4_7[1,0], \; R^4_7[0,0] ) $\;
			$articular[5] = Atan2(\sqrt{1- (R^4_7[2][0])^2}, \; R^4_7[2,0] ) $\;
			$articular[6] = Atan2(R^4_7[2,2] ,\;  -R^4_7[2,1] ) $\;

			\If{ $articular[4] > \pi/2$}
			{
				$articular[4] -= \pi$ \;
				$articular[5] *= -1$ \;
			}
			\ElseIf{$articular[4] < -\pi/2$}
			{
				$articular[4] += \pi$ \;
				$articular[5] *= -1$ \;
			}

			\BlankLine
			\BlankLine
			\BlankLine

			$\psi_1 = atan2(\frac{R_4^7[2][2]}{-R_4^7[2][1]})$\; 
			$\psi_2 = atan2(\frac{-R_4^7[2][2]}{R_4^7[2][1]})$\; 

			\If{ $fabs(\psi_1) < fabs(\psi_2)$}
			{
				$articular[6] = \psi_1$ \;
			}
			\Else
			{
				$articular[6] = \psi_1$ \;
			}	

		}

			
		\BlankLine
		\BlankLine
		\BlankLine

		\textbf{return} $articular[]$ \;
		\caption{Condiciones para el cálculo de la cinemática inversa por el método geométrico.}
		\label{Alg:IKgeome}
		\end{algorithm}\DecMargin{1em}


	Uno de los objetivos de este trabajo consiste en realizar una comparación entre este algoritmo y un resolvedor de cinemáticas estándar el cual utiliza un método iterativo para tal propósito. Tal paquete es nombrado \textit{moveIt!} y es compatible con la paquetería ROS.\\ 

	\section{Cinemática inversa paquetería MoveIt}

	La paquetería de software \textit{moveIt!} consiste en una serie de librerías enfocadas a las tareas de manipulación de objetos en la robótica. Paquetes como \textit{arm\_navigation} fueron diseñados con el objetivo especifico de ayudar en tareas de planeación de movimientos, generación de trayectorias y monitoreo del ambiente particularmente para los manipuladores del robot PR2 \cite{chitta2012moveit}. La idea original de la paquetería moveIt! es generar planes de manipulación y trayectorias basándose información de un modelo del entorno. Dichos modelos del entorno se crean, comúnmente, utilizando la fusión de datos del sensor láser y de sensores estéreo colocados en los robots.\\


	El ambiente es representado como una mezcla de datos proporcionados en dos formatos:\\

	\begin{enumerate}
		\item{ Una red voxelizada que representa la mayor parte de los obstáculos en el medio ambiente. }

		\item{ Primitivas geométricas y modelos de mallas para representar objetos que han sido reconocidos y registrado en el medio ambiente mediante rutinas de detección de objetos. }
	\end{enumerate}

	De esta manera la información del modelo del entorno sirve de entrada principal a los planificadores que constituyen la paquetería \textit{moveIt}.\\


	\textit{MoveIt!} incorpora herramientas para planeación de movimientos, cinemática, percepción 3D, control y navegación. Además provee una plataforma de uso fácil para desarrollo de aplicaciones robóticas avanzadas, permitiendo la evaluación de nuevos diseños de robots y la construcción de productos robóticos para su uso industrial, comercial y de investigación, entre otros. \textit{MoveIt!} Es el software de código abierto más usado para manipulación de robots \cite{chitta2012moveit}.\\


	En la actualidad \textit{moveIt} provee una interfaz genérica para la planeación de movimiento de manipuladores que puede ser integrada muy fácilmente en ROS. Esta interfaz permite la integración de diferentes tipos de planeadores de movimientos, por ejemplo:\\

	\begin{enumerate}
		\item{Planeadores aleatorizados. Open Motion Planning Library (OMPL).}

		\item{Planeadores basados en búsquedas. (SBPL)}

		\item{Librerías de optimización de trayectorias. (CHOMP)}

		\item{Librerías de optimización estocástica para la planeación de movimientos. (STOMP)}
	\end{enumerate} 

	La paquetería \textit{moveIt} permite personalizar el solucionador de cinemáticas para realizar cálculos más rápidos de la cinemática inversa, utilizando para ello métodos numéricos. Una característica importante de la paquetería \textit{moveIt} es que para solucionar la cinemática inversa extrae de la información necesaria desde un archivo URDF (archivo estándar para descripción de robots en el formato aceptado por ROS).\\ 

	De esta manera se utilizó el archivo URDF que contiene la información con la descripción de las transformaciones entre los elementos del robot Justina. Para ello se creó un archivo extra que contiene unicamente la información de traslaciones entre los elementos que constituyen los brazos del robot Justina. Como se muestra en la figura \ref{fig:desciption_arms1}.\\

	\begin{figure}[H]
	\centering
	\begin{subfigure}[t]{.55\textwidth}
		\centering
		\includegraphics[width=3.50cm, height=7.0cm]{manipulator/DH_manipulators1.png}
		\caption{URDF descripción del robot Justina.}
	\end{subfigure}%
	\begin{subfigure}[t]{.55\textwidth}
		\centering
		\includegraphics[width=3.50cm, height=6.0cm]{manipulator/moveIt_manipulators2.png}
		\caption{URDF descripción de las transformaciones\\ de los brazos del robot Justina.}
	\end{subfigure}
	\caption{Vista trimétrica del despliegue del robot Justina y sus respectivos manipuladores en el visualizador gráfico Rviz.}
	\label{fig:desciption_arms1}
	\end{figure}


	\begin{figure}[H]
	\begin{subfigure}[t]{.55\textwidth}
		\centering
		\includegraphics[width=3.50cm, height=7.0cm]{manipulator/DH_manipulators2.png}
		\caption{URDF descripción del robot Justina.}
	\end{subfigure}%
	\begin{subfigure}[t]{.55\textwidth}
		\centering
		\includegraphics[width=3.50cm, height=6.0cm]{manipulator/moveIt_manipulators1.png}
		\caption{URDF descripción de las tranformaciones\\ de los brazos del robot Justina.}
	\end{subfigure}
	\caption{Imagen de la comparación entre descripciones de los brazos del robot Justina.}
	\label{fig:desciption_arms2}
	\end{figure}

	Con la información del archivo URDF se creó un nuevo archivo que contiene la información de las relaciones entre los elementos que constituyen al robot Justina, tal documento es conocido como SURDF, por ser una descripción semántica del robot.\\

	Una vez que se generaron tales archivos se procedió a crear un modelo cinemático de los brazos del robot Justina. Tal modelo cinemático de los brazos nos da acceso a funciones para obtener la cinemática inversa, utilizando el solucionador de cinemática KLD integrado en la paquetería \textit{moveIt!}.\\

	Esta sección en particular tiene como objetivo comparar ambos métodos de solución de la cinemática inversa del robot de servicio Justina. Por tanto, en el presente trabajo se analiza y compara el tiempo de ejecución de ambos algoritmos. Y como característica extra se plantea una función de utilidad para evaluar la \textit{conveniencia} de utilizar un algoritmo u otro.\\

	\begin{figure}[h!]
		\begin{center}
		\includegraphics[scale=0.30]{ros/moveIt_structure.jpg}	
		\caption{Imagen de la estructura de la paquetería \textit{moveIt!}.\cite{moveitEpage}}
		\end{center}
	\end{figure}

	\section{Planteamiento de la función de costo para el cálculo de la cinemática inversa.}

	Al resolver el problema de la cinemática inversa se pueden obtener múltiples soluciones para una misma información de entrada $x, y, z, roll, pitch, yaw$. Por ello, es importante determinar en qué casos es conveniente tomar la información obtenida mediante un método geométrico o mediante un método iterativo. Por tal razón en el presente trabajo de tesis se plantea una función de costo que nos permita comparar y cuantificar ambos algoritmos de solución de cinemáticas.\\

	Para el planteamiento de la función de costo se tomaron en cuenta los siguientes aspectos:\\

	\begin{itemize}
		\item{Los valores de ángulos más alejados de cero implican un mayor gasto energético de los motores.}
		\item{Los ángulos de los motores 0 y 5, deben ser negativos para obtener un configuración de codo abajo y conservar una estructura antropomórfica de los brazos.}
	\end{itemize}

	\begin{equation}
	\begin{split}
		\epsilon = f(\theta_0, \theta_1, \theta_2, \theta_3, \theta_4, \theta_5, \theta_6) \cr
		\epsilon =  \sum_{i=0}^n |\theta_i| * \alpha
	\end{split}
	\label{Ec:funcCost}
	\end{equation}

	Donde $\alpha$ representa una ganancia que establecimos con un valor de 0.5. Para corregir el problema con la morfología de la toma de objetos se optó por restringir los valores de ángulos dentro del archivo que contiene la descripción del robot. De esta manera se realizaron las pruebas llamando dos servicios para el cálculo de la cinemática inversa y evaluando la función de costo con la respuesta obtenida.\\


	\begin{figure}[H]
		\centering
		\includegraphics[width=11.00cm, height=5.5cm]{ros/cinematicaServices1.png}
		\caption{Imagen representativa de la comunicación entre módulos para evaluar la función de costo propuesta para el cálculo de la cinemática inversa.}
		\label{fig:cinematica_services}
	\end{figure}

	Los resultados de realizar esta evaluación se reportan en el capítulo 6.  



\newpage
\thispagestyle{empty}
\mbox{}
%\addtocounter{page}{-1}