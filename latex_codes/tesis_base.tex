\documentclass[12pt]{report}
\usepackage[utf8]{inputenc}
\usepackage{graphicx}
\graphicspath{ {images/} }

\title{
	{Caracterización de un sistema de reconocimiento y manipulación de objetos par un robot de servicio}\\
	{\large Universidad Nacional Auntónoma de México}\\
}


\author{Edgar de Jesús Vázquez Silva}
\date{Noviembre 2016}

\chapter*{Introducción}
	\begin{itemize}
		\item ¿Por qué son necesarios los robots de servicio?

		\item Una de las tareas más requeridas en un robot de servicio es la manipulación de objetos
		por ello es necesario... 

		\item Plantear la problemática

		\item Hipotesis

		\item Objetivos

		\item Descripción del documento 

	\end{itemize}



\chapter*{Marco téorico}
	\begin{itemize}
		\item Los robots de servicio 

		\item Las técnicas de roconocimiento de objetos

		\item Cinemática utilizadas en brazos robóticos

		\item Planeación de tares

	\end{itemize}



\chapter*{Reconocimiento de objetos}
	\begin{itemize}
		\item Kinect 
		\item OpenCV
	\end{itemize}



\chapter*{Brazos}
	\begin{itemize}
		\item Documentación servos dynamixel
		\item Cinemática inversa
		\item Propuesta de algoritmo para mejorar la cinemática inversa (Algoritmo genético)
	\end{itemize}



\chapter*{Integración}
	\begin{itemize}
		\item Máquinas de estados
		\item ROS
		\item Constitución del robot de servicio Justina
	\end{itemize}



\chapter*{Resultados}



\chapter*{Conclusiones}


\tableofcontents



